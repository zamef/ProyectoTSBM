\documentclass{replab}
\usepackage{lipsum}

% --- Información del documento ---
\title{Proyecto}
\author{Zahid Medrano Flores}
\subtitle={Detección y clasificación de Queratocono}
\subject={\textit{Temas Selectos en Biomatemáticas.} \textbf{Introducción a la Ciencia de Datos.}}

\setlength{\columnsep}{14pt}

% --- Archivo de bibliografía ---
\addbibresource{ref.bib}

% --- Inicio del documento ---
\begin{document}
	
	\pagestyle{fancy}
	\unspacedoperators
	
% --- Título ---
\selectlanguage{spanish}
	
% --- Cuerpo del reporte ---

\twocolumn[
    \begin{center}
        \maketitle

            {\begin{tcolorbox}[colframe=white, colback=principaldos, arc=6pt]
				\begin{onecolabstract}

                    \noindent El queratocono es un padecimiento que afecta a la córnea, produciendo un empeoramiento en la visión de la persona. El objetivo principal de este proyecto es abordar la detección temprana y oportuna del Queratocono.
                    
                    \noindent Se cuenta con un dataset de aproximadamente 3200 registros y con los cuales podemos intentar predecir si se tiene o no el padecimiento, así como su posible etapa. Todo esto con diversas técnicas de Machine Learning y ciencia de datos en general.
    
                \end{onecolabstract}
			\end{tcolorbox}}
    \end{center}
]

\bigskip
	
\section{Introducción}



\section{Desarrollo}



\section{Resultados}



\section{Análisis y Discusión}



% ----------------------------------------------------------------

\DeclareFieldFormat{labelnumberwidth}{\mkbibbrackets{#1}}
\defbibenvironment{bibliography}
  {\list
     {\printtext[labelnumberwidth]{%
      \printfield{labelprefix}%
      \printfield{labelnumber}}}
     {\setlength{\labelwidth}{\labelnumberwidth}%
      \setlength{\leftmargin}{\labelwidth}%
      \setlength{\labelsep}{\biblabelsep}%
      \addtolength{\leftmargin}{\labelsep}%
      \setlength{\itemsep}{\bibitemsep}%
      \setlength{\parsep}{\bibparsep}}%
      \renewcommand*{\makelabel}[1]{\hss##1}}
  {\endlist}
  {\item}

\printbibliography[heading=bibintoc]

\end{document}